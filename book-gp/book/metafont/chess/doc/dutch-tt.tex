%
% dutch-tt.tex
% ----------------
% A plain TeX torture test for the chess.sty style in combination with 
% dutch.sty. 
% Author : Piet Tutelaers (internet: rcpt@urc.tue.nl)
% Version: 1.2 (June 1991)
%
\input dutch.sty
\input chess.sty

\tracinglostchars=1

{\bf Torture test for {\tt chess.sty}}

Check if language support for Dutch is working:
\board{tpldklpt}
      {iiiiiiii}
      { * * * *}
      {* * * * }
      { * * * *}
      {* * * * }
      {IIIIIIII}
      {TPLDKLPT}

$$\showboard$$

Let's define a position:
\position
\White(Ke1,Th1,h7,b2)
\Black(Ke8,Ta8,a5,c2)
\global\movecount=50
\global\Whitefalse
\endposition

$$\showboard$$

The castling test:
\ply         e8c8!?
\ply  e1g1

Both players have castled.
\ply         c8e8?
\ply  g1e1?

They move back their kings... Chess.sty does not check the
validity of moves.
\ply         e8c8
\ply  e1g1??

But now these moves are not typeset as castling moves because the kings have
moved.

And now the `en passant' test:
\ply         a5a4
\move b2b4+  a4b3ep

A short analysis test: |54. b3+ a*b3| and White looses his pawn.
Better seems |54.Pf7 but not 54.Pg6 because then 54:Tg8!|.
And the promotion test (comment thanks to Jaap Sprey!):

\move  h7h8	c2c1L!?

$$\showboard$$

Activation of analysis mode does not work inside arguments of macros but
you can group them around the macro!

|\centerline{55.Pf3 Pc6}|

Sometimes you want to use the {\tt \char124} character for other
purposes than to trigger the analysis mode.  Use then the {\tt nochess}
environment.  Within this environment you can reactivate the analysis
mode with the {\tt chess} environment. 

\nochess
|55.Pf3 Pc6|

\chess
\centerline{55.Pf3 Pc6}
\endchess

|55.Pf3 Pc6|
\endnochess

|KRQDRTBLFPCS| that's it!
\bye

