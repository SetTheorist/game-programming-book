% File:	       tal.tex
% Version 1.2: Febr. 1992

% Changes from version 1.1
% A number of typos corrected.  When I replayed this game I discovered a
% winning position for Fisher (after move 50 ...  B-R8 in Fisher's book:
% My 60 Memorable Games).  But this turned out to be a typo from Fisher
% that was corrected in his later book (Bobby Fisher: Fisher's chess games;
% Oxford chess books, 1980, IBN 0-19-217566-1). The typo was copied also
% in Bijl (Die Gesammelte Partien, pp. 177) and Euwe (Fisher en zijn
% voorgangers, pp 280).

\input chess.sty
\hsize=6.2truein
\parindent=0pt
\overfullrule=0pt
\input 2side
\gutter{20pt}
\font\sc=cmcsc10

{\bf Fischer -- Tal}\smallskip
Candidates' Tournament 1959\medskip
{\sl A very near miss\smallskip
This is one of the four games that Fisher lost to Tal who, in winning
this tournament, earned the right to meet and trounce Botvinnik for the
world championship.

In jest the whimsical Tal signed Fisher's name, in addition to his own,
when asked for an autograph. ``Why not?'' he quipped; ``I've beaten
Bobby so often |:| that gives me the right to sign for him!''

A carefull reading of Fisher's notes will reveal a clear echo of the
strong emotions that engulfed him during this tense encounter. He misses
a win in the opening and several draws along the way, demonstrating
dramatically how a continuously advantageous position can abruptly be
turned into defeat by seemingly insignificant miscalculations.}

{\sc Sicilian Defense}\medskip

\newgame
\move e2e4 c7c5
\move g1f3 d7d6
\move d2d4 c5d4
\move f3d4 g8f6
\move b1c3 a7a6
\move f1c4 e7e6
\ply  c4b3

I had no better luck against Blackstone, in an exhibition game at Davis,
California, 1964, with |7. 0-0, Be7; 8. Bb3, Qc7; 9. f4, b5; 10. f5, b4;
11. f*e6!? (11. Nce2, e5; 12. Nf3, Bb7| is bad for White), |b*c3; 12. e*f7+, 
Kf8; 13. Bg5, Ng4!| and Black should win.

\ply       b7b5!

This reaction must be prompt! 

In our first lap game here Tal played the weaker |7.: Be7?; 8. f4, 0-0|
(for |8.: b5| see the note to Black's 8th move); |9. Qf3, Qc7 and now
10. f5! (instead of 10. 0-0?, b5; 11. f5, b4!; 12. Na4, e5; 13. Ne2,
Bb7| and Black stands better), |e5 (not 10.: Nc6; 11. Be3 with a bind);
11. Nde2, b5; 12. a3, Bb7; 13. g4| with a strong attack.

\move f2f4!? b5b4!
Indirectly undermining White's center.

\move c3a4 f6e4
\move e1g1 g7g6?
Correct is |10.: Bb7|.

\ply  f4f5!
This riposte caught Tal completely unaware. Black's king trapped in the
center, will soon be subject to mayhem.

\ply       g6f5
Not |11.: e*f5; 12. Bd5, Ra7; 13. N*f5!, g*f5; 14. Qd4|.

\ply  d4f5!
Panov, with typical iron curtain ``objectivity'', commented in the
Soviet tournament bulletins: ``Almost all game Fischer played in Tal
style. But all his trouble was in vain because Tal did not defend in
Fischer style---instead he found the one and only saving
counterchance!'' 

$$\showboard$$

\ply       h8g8
Woozy, Tal stumbles into a dubious defense. Better is |12.: d5 (not 12.:
e*f5?; 13. Qd5, Ra7; 14. Qd4 spearing a rook); 13. Nh6, B*h6; 14. B*h6|.

\ply  b3d5!
A shot!

\ply       a8a7
``|13.: e*d5; 14. Q*d5, B*f5; 15. R*f5, Ra7; 16. Qe4+; Re7; 17. Q*b4,
Re2; 18. Bg5!, R*g5; 19. R*g5, Q*g5; 20. Q*b8+ wins|'' ({\sc PANOV}).  

\ply  d5e4?
Correct is |14. Be3!, Nc5; 15. Qh5!, Rg6 (if 15.: N*a4, 16. B*a7, e*d5;
17. Rae1+); 16. Rae1!| and White's every piece is bearing down on Black's
king ({\sc KEVITZ}).

\ply       e6f5
\ply  e4f5
Probably it's better to avoid exchanges with |15. Bd5 or Bf3|.

\ply       a7e7!
A unique way of shielding the K-file.

\move f5c8 d8c8
\ply  c1f4?
The right move is simply |17. c3! (not 17. Q*d6?, R*g2+; 18. K*g2, Re2+;
19. Kf3, B*d6; 20. K*e2, Q*c2+ wins), and if : Qc6; 18. Rf2|.

\ply       c8c6!
\move d1f3 c6a4
Such a surprise that I didn't dare believe my eyes! I had expected |18.:
Q*f3; 19. R*f3, Re2; 20. Rf2, R*f2; 21. K*f2| and White has a slight
edge after a3 because of Black's disconnected pawns.

\move f4d6 a4c6!
Tal finds a inspired defense.

\move d6b8 c6b6+
White remains a clear pawn ahead after |20.: Q*f3; 21. R*f3, Bg7; 22. c3|.

\move g1h1 b6b8
The crowd was shouting and whistling with each move. Later I was
informed that many sport fans were in the audience. Maybe some soccer
match had been canceled. As a consequence chess was the main attraction 
that day in Belgrade.

\ply  f3c6+
Many annotators believed that |22. Rae1| was the winning move. Tal
himself confessed he thougt Black was lost after that. But |22.: Kd8!|
holds in all lines (not |22.: Rg6?; 23. Q*f7+, Kd7; 24. Rd1+!, Rd6; 25.
R*d6+, K*d6; 26. Rf6+!| wins). I've studied this position for ages, it
seems, and the best I can find is |23. Rd1+, Kc7! (23.: Kc8?; 24. Qc6+
wins); 24. Qf4+ (if 24. Rd4, Qb7!), Kb7; 25. Rd6, Qc7; 26. Q*b4+, Kc8;
27. R*a6, Qb7!; 28, Q*b7+, K*b7; 29. Raf6, Rg7|=.

\ply       e7d7
\ply  a1e1+
Black holds after |23. Rad1, Bd6; 24. R*f7 (if 24. Rf6, Rg6; 25. R1*d6?,
Q*d6!), Qc7,| etc. And on |23. R*f7, Qd6|.

\ply       f8e7
Finally Tal ``develops'' his bishop. Not |23.: Kd8; 24. R*f7!, Be7; 25.
Rf*e7, R*e7; 26. Rd1+| wins.

\move f1f7 e8f7
\move c6e6+ f7f8!
I thought he had to go to g7, whereupon |26. Q*d7| wins easily. 

$$\showboard$$

\ply  e6d7
Not |26. Rf1+, Kg7; 27. Rf7+, Kh8; and if 28. Q*d7, Rd8; 29. Qg4, Qe5|
wins. 

\ply       b8d6
\move d7b7 g8g6
Within a handful of moves the game has changed its complexion. Now
it is White who must fight for a draw!

\ply  c2c3!
Black's extra piece means less with each pawn that's exchanged.

\ply       a6a5
On |28.: b*c3; 29. Qc8+, Bd8; 30. Q*c3|=. 

\ply  b7c8+
On the wrong track. Right is |29. c*b4!, Q*b4 (if 29.: a*b4; 30. a3!,
b*a3; 31. b*a3, Q*a3 draws); 30. Qf3+, Kg7; 31. Qe2| draws, since Black
can't possibly build up a winning K-side attack and his own king is to
exposed.  

\ply       f8g7
\move c8c4 e7d8
\move c3b4 a5b4
On |31.: Q*b4; 32. Qe2| White should draw with best play.

\ply  g2g3?
Creating loosing chances. I don't see how Black can make any progress
after |32. Qe4|. If |32.: Bc7; 33. Qe7+, Kg8; 34. Qe8+, Qf8; 35. Qe4|,
etc. 

\ply       d6c6+
\move e1e4 c6c4
\move e4c4 g6b6!
I overlooked this. Now Black has winning chances. I had planned on a
draw after |34.: Be7?; 35. a3!| dissolving Black's b--pawn (|35.: b3 is
answerred by 36. Rc7 followed by Rb7|).

\move h1g2 g7f6
\move g2f3 f6e5
\ply  f3e3
|37. a3| is met as always, by b3. Once White can eliminate Black's
b--pawn it's a theoretical draw.

\ply       d8g5+
\move e3e2 e5d5
\move e2d3 g5f6
White might be able to draw this ending, but it's an ugly defensive chore.

\ply  c4c2?
Too passive. I wanted to avoid immobilizing my Q--pawns with |40. b3|
but it's the best hope now. On |40.: Be7; 41. Rd4+| preserves drawing
chances. 

$$\showboard$$

\ply       f6e5
\move c2e2 b6f6
\move e2c2 f6f3+
\move d3e2 f3f7
\move e2d3 e5d4!
Little by little Tal inches his way in.

\ply  a2a3
On |45. b3, Rf3+; 46. Ke2, Rf2+; 47. Qd3, R*c2; 48. Ke4| wins.

\ply       b4b3
\ply  c2c8
Equally hopeless is |46. Re2 (or 46. Rd2, Rf3+; 47. Ke2, Rf2+), Rf3+;
47. Kd2, B*b2| etc.

\ply       d4b2
\move c8d8+ d5c6
\move d8b8 f7f3+
\move d3c4 f3c3+
\move c4b4 c6c7 % c6c7 and b8b5 missing in Fisher's book!
(Here the author omits |50: Kc7; 51. Rb5| but overlooked that he could
win the game {(\it pt})!)

\move b8b5 b2a1
\move a3a4 b3b2! 

$$\showboard$$

White resigns.

If |53. K*c3, b1Q+!|

From: {\sl My 60 Memorable Games}, by Bobby Fischer; Faber and Faber,
London. 1969. {\sc ISBN} 0 571 09312 4
\bye
